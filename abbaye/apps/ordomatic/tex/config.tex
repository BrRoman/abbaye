\documentclass[twoside]{article}

% Géométrie de la page :
\usepackage{geometry}
\geometry{paperwidth=10.5cm, paperheight=16.5cm, inner=0.75cm, outer=0.75cm, tmargin=0.5cm, bmargin=0.5cm, includehead}

% Choix d'une police principale :
\usepackage{fontspec}
\setmainfont{Garamond Premier Pro}
% Fonts configuration file:
\newfontfamily{\FontFlyLeaf}[Path=/usr/share/fonts/opentype/Silentium Pro/]{SilentiumPro-Roman I}

% Césures etc. :
\usepackage[latin]{babel}
\usepackage{array}
\usepackage{enumitem}

% Mise en forme des titres de sections (? Janvier, février etc. ?) :
\usepackage[explicit]{titlesec}
\titleformat{\section}{}{}{0cm}{\center\MakeUppercase{#1}}

% Entêtes et pieds de pages :
\usepackage{fancyhdr}
\pagestyle{fancy}
\fancyhf{}
\fancyhead[RO,LE]{\thepage}
\fancyhead[CE]{\textbf{\rightmark}}
% Le [CO] étant variable sera défini dans le fichier maître par programmation.
\renewcommand{\sectionmark}[1]{\markright{#1 \CurrentYear{}}}
\renewcommand{\headrulewidth}{0pt}
\renewcommand{\footrulewidth}{0pt}
\setlength{\parindent}{0cm}
\setlength{\headsep}{0.3cm} % Distance entre le header ("Cyclus liturgicus etc.") et le corps du texte.

% Tableaux spéciaux sur plusieurs pages :
\usepackage{longtable}

% Pour forcer l'usage des césures (cf. mail Thierry Masson), ATTENTION DANGER (entêtes et autres blocs bizarres) :
\pretolerance = -1
\tolerance = 2000

% Pour augmenter l'approche des caractères :
\usepackage{soul}



%%%%%%%%%%%%%%%%%%%%%%%%%%%%%%%
% Commandes et environnements :
%%%%%%%%%%%%%%%%%%%%%%%%%%%%%%%

%% Espace fine :
\DeclareRobustCommand{\mynobreakthinspace}{%
    \leavevmode\nobreak\hspace{0.08em}}
\def~{\mynobreakthinspace{}}

% Environnement Boîte (utile chaque fois qu'on veut commencer par un espace vertical avant le paragraphe) :
\newenvironment{ParBox}[2]
{
    \setlength{\parindent}{0cm}
    \begin{center}
        \parbox{8.5cm}{\vspace{#1} #2}}
        {
    \end{center}
    \par}
\newcommand{\ApplyParBox}[2]{\begin{ParBox}{#1}{#2}\end{ParBox}}

% Style de paragraphe NewMonthTitres :
\newenvironment{ParStyleNewMonthTitles}[1]
{
    \setlength{\parskip}{0cm}
    \fontsize{14}{13}\selectfont
    \bfseries
    \section{#1}}{}
\newcommand{\ApplyNewMonthTitles}[1]{\begin{ParStyleNewMonthTitles}{#1}\end{ParStyleNewMonthTitles}}

% Style de paragraphe NewMonthSousTitres :
\newenvironment{ParStyleNewMonthSubTitles}[1]
{
    \fontsize{12}{13}\selectfont
    \bfseries
    \itshape
    \begin{center}
        \MakeUppercase{#1}}
        {
    \end{center}}
\newcommand{\ApplyNewMonthSubTitles}[1]{\begin{ParStyleNewMonthSubTitles}{#1}\end{ParStyleNewMonthSubTitles}}

% Commande nouveau mois pour entêtes :
\newcommand{\NewMonthForHeader}[1]{\renewcommand{\leftmark}{\normalfont{#1}}}

% Style de paragraphe GeneralitiesTitres1 :
\newenvironment{ParStyleGenerTitleHuge}[1]
{
    \fontsize{15}{19}\selectfont
    \setlength{\parskip}{-0.5cm}
    \begin{center}
        \MakeUppercase{\textbf{#1}}}
        {
    \end{center}}
\newcommand{\ApplyGenerTitleHuge}[1]{\begin{ParStyleGenerTitleHuge}{#1}\end{ParStyleGenerTitleHuge}}

% Style de paragraphe GeneralitiesTitres2 :
\newenvironment{ParStyleGenerTitleLarge}[1]
{
    \fontsize{12}{14.5}\selectfont
    \setlength{\parskip}{-0.5cm}
    \begin{center}
        \MakeUppercase{\textbf{#1}}}
        {
    \end{center}}
\newcommand{\ApplyGenerTitleLarge}[1]{\begin{ParStyleGenerTitleLarge}{#1}\end{ParStyleGenerTitleLarge}}

% Style de paragraphe GeneralitiesSousTitres :
\newenvironment{ParStyleGenerSubTitle}[1]
{
    \fontsize{11.5}{14}\selectfont
    \begin{center}
        \textbf{#1}}
        {
    \end{center}
    \par}
\newcommand{\ApplyGenerSubTitle}[1]{\begin{ParStyleGenerSubTitle}{#1}\end{ParStyleGenerSubTitle}}

% Style de paragraphe GeneralitiesListes :
\newenvironment{ParStyleGenerList}
{
    \fontsize{9.5}{10.5}\selectfont
    \setlength{\parskip}{0.5cm}
    \begin{itemize}
        [label = -,
            labelwidth = 0.3cm,
            partopsep = -0.5cm,
            itemsep = 0cm,
            leftmargin = 0.5cm]}{
    \end{itemize}}
\newcommand{\ApplyGenerList}[1]{\begin{ParStyleGenerList}#1\end{ParStyleGenerList}}

% Style de paragraphe Anniv :
\newenvironment{ParStyleAnniv}
{
    \setlength{\parindent}{0cm}
    \setlength{\parskip}{0.5cm}
    \fontsize{9.5}{10.5}\selectfont
    \itshape}
{
    \par}
\newcommand{\ApplyAnniv}[1]{\begin{ParStyleAnniv}#1\end{ParStyleAnniv}}

% Style de paragraphe HebdoPsalt :
\newenvironment{ParStyleHebdoPsalt}
{
    \fontsize{11.5}{13}\selectfont
    \setlength{\parindent}{0cm}
    \setlength{\parskip}{0cm}
    \itshape
    \begin{center}}
        {
    \end{center}
    \par}
\newcommand{\ApplyHebdoPsalt}[1]{\begin{ParStyleHebdoPsalt}\vspace{.3cm}#1\vspace{-.3cm}\end{ParStyleHebdoPsalt}}

% Style de paragraphe Header :
\newenvironment{ParStyleHeader}
{
    \fontsize{11}{12}\selectfont
    \setlength{\parindent}{0cm}
    \setlength{\parskip}{0.3cm}
    \setlength{\tabcolsep}{0cm}
    \begin{tabular}{p{0.7cm} p{8.3cm}}}
        {
    \end{tabular}
    \par}
\newcommand{\ApplyHeader}[1]{\begin{ParStyleHeader}#1\end{ParStyleHeader}}

% Style de paragraphe Body :
\newenvironment{ParStyleBody}
{
    \fontsize{10}{11.5}\selectfont
    \begin{itemize}
        [label = -,
            labelwidth = 0.3cm,
            partopsep = 0cm,
            itemsep = 0cm,
            leftmargin = 0.7cm]}
        {
    \end{itemize}}
\newcommand{\ApplyBody}[1]{\begin{ParStyleBody}#1\end{ParStyleBody}}

% Style de paragraphe LecturesHeader :
\newenvironment{ParStyleLectHeader}
{
    \fontsize{10}{11}\selectfont
    \setlength{\parindent}{0cm}
    \setlength{\parskip}{0cm}
    \bfseries
    \begin{center}}
        {
    \end{center}
    \par}
\newcommand{\ApplyLectHeader}[1]{\begin{ParStyleLectHeader}#1\end{ParStyleLectHeader}}

% Style de paragraphe LecturesBody :
\newenvironment{ParStyleLectBody}
{
    \fontsize{9}{10.5}\selectfont
    \begin{list}{}{
        \setlength{\labelwidth}{1.2cm}% largeur de la boîte englobant le label
        \setlength{\labelsep}{0.1cm}% espace entre paragraphe et l’étiquette
        \setlength{\leftmargin}{1.8cm}% marge de gauche - Normalement \labelwidth+\labelsep, mais ça marche pas. - En pratique mettre \labelwidth + \labelsep + la \leftmargin du Style Body. Ainsi les lectures sont alignées sur les items des paragraphes Body.
        \setlength{\itemsep}{-0.05cm}% espace entre les items.
              \renewcommand{\makelabel}[1]{\raggedleft{##1}\hfill}}}
              {
    \end{list}}
\newcommand{\ApplyLectBody}[1]{\begin{ParStyleLectBody}#1\end{ParStyleLectBody}}

% Style de paragraphe LecturesTriduum :
\newenvironment{ParStyleLectTrid}
{
    \fontsize{9}{10.5}\selectfont
    \setlength{\parskip}{-0.2cm}}
{}
\newcommand{\ApplyLectTriduum}[1]{\begin{ParStyleLectTrid}#1\end{ParStyleLectTrid}}

% Style de paragraphe Préface féries :
\newenvironment{ParStylePrefaceFeries}
{
    \fontsize{9}{10.5}\selectfont
    \begin{itemize}
        [label = -,
            labelwidth = 0.3cm,
            partopsep = 0cm,
            itemsep = 0cm,
            leftmargin = 0.5cm]}
        {
    \end{itemize}}
\newcommand{\ApplyPrefaceFeries}[1]{\begin{ParStylePrefaceFeries}#1\end{ParStylePrefaceFeries}}


% Symboles spéciaux :
% Abstinence :
\catcode`\ł=\active
\defł{{\fontspec{Menlo} \symbol{10063}}}
% Jeûne et abstinence :
\catcode`\µ=\active
\defµ{{\fontspec{Menlo} \symbol{9724}}}
% 1er vendredi du mois :
\catcode`\£=\active
\def£{{\fontspec{Menlo} \symbol{10070}}}
% 1er samedi du mois :
\catcode`\§=\active
\def§{{\fontspec{Menlo} \symbol{10037}}}
% 1er dimanche du mois :
\catcode`\ŧ=\active
\defŧ{{\fontspec{Menlo} \symbol{10016}}}
% Fête précepte :
\catcode`\¬=\active
\def¬{{\fontspec{Menlo} \symbol{9670}}}
% Fête demi-précepte :
\catcode`\þ=\active
\defþ{{\fontspec{Menlo} \symbol{10061}}}
% Antienne :
\catcode`\ø=\active
\defø{{\fontspec{FlavGaramond} \symbol{8721}}}
% Répons :
\catcode`\¶=\active
\def¶{{\fontspec{FlavGaramond} \symbol{164}}}
% Verset :
\catcode`\ß=\active
\defß{{\fontspec{FlavGaramond} \symbol{8730}}}
% Croix :
\catcode`\†=\active
\def†{{\fontspec{FlavGaramond} \symbol{8224}}}

