\vspace{0.2cm}

\begin{paracol}{2}

    \LigneParacol{0cm}
    {Te ígitur, clementíssime Pater, per Iesum Christum, Fílium tuum, Dóminum nostrum, súpplices rogámus ac pétimus, uti accépta hábeas et benedícas hæc dona, hæc múnera, hæc sancta sacrifícia illibáta.}
    {Père infiniment bon, toi vers qui mon\-tent nos louanges, nous te supplions par Jésus Christ, ton Fils, notre Seigneur, d'accepter et de bénir ces offrandes saintes.}

    \LigneParacol{0cm}
    {In primis, quæ tibi offérimus pro Ecclésia tua sancta cathólica~: quam pacificáre, custodíre, adunáre et régere dignéris toto orbe terrárum~: una cum fámulo tuo Papa nostro → et Antístite nostro → et ómnibus orthodóxis atque cathólicæ et apostólicæ fídei cultóribus.}
    {Nous te les présentons avant tout pour ta sainte Église catholique~: accorde-lui la paix et protège-la, daigne la rassembler dans l'unité et la gouverner par toute la terre~; nous les présentons en même temps pour ton serviteur le Pape →, pour notre évêque → et tous ceux qui veillent fidèlement sur la foi catholique reçue des Apôtres.}

    \LigneParacol{0cm}
    {Meménto, Dómine, famulórum famularúmque tuárum → et → et ómnium circumstántium, quorum tibi fides cógnita est et nota devótio, pro quibus tibi offérimus~: vel qui tibi ófferunt hoc sacrifícium laudis, pro se suísque ómnibus~: pro redemptióne animárum suárum, pro spe salútis et incolumitátis suæ~: tibíque reddunt vota sua ætérno Deo, vivo et vero.}
    {Souviens-toi, Seigneur, de tes serviteurs (de → et →) et de tous ceux qui sont ici réunis, dont tu connais la foi et l'attachement. Nous t'offrons pour eux, ou ils t'offrent pour eux-mêmes et tous les leurs ce sacrifice de louange, pour leur propre rédemption, pour le salut qu'ils espèrent~; et ils te rendent cet hommage, à toi, Dieu éternel vivant et vrai.}

    \LigneParacol{0cm}
    {Communicántes, et memóriam venerántes, in primis gloriósæ semper Vírginis Maríæ, Genetrícis Dei et Dómini nostri Iesu Christi~: sed et beáti Ioseph, eiúsdem Vírginis Sponsi, et beatórum Apostolórum ac Mártyrum tuórum, Petri et Pauli, Andréæ, Iacóbi, Ioánnis, Thomæ, Iacóbi, Philíppi, Bartholomǽi, Matthǽi, Simónis et Thaddǽi~; Lini, Cleti, Cleméntis, Xysti, Cornélii, Cypriáni, Lauréntii,Chrysógoni, Ioánnis et Pauli, Cosmæ et Damiáni et ómnium Sanctórum tuórum~; quorum méritis precibúsque concédas, ut in ómnibus protectiónis tuæ muniámur auxílio. Per Christum Dóminum nostrum. Amen.}
    {Dans la communion de toute l'Église, nous voulons nommer en premier lieu la bienheureuse Marie toujours Vierge, Mère de notre Dieu et Seigneur, Jésus Christ~; saint Joseph, son époux, les saints Apôtres et Martyrs Pierre et Paul, André, Jacques et Jean, Thomas, Jacques et Philippe, Barthélemy et Matthieu, Simon et Jude, Lin, Clet, Clément, Sixte, Corneille et Cyprien, Laurent, Chrysogone, Jean et Paul, Côme et Damien et tous les saints. Accorde-nous, par leur prière et leurs mérites, d'être, toujours et partout, forts de ton secours et de ta protection.}

    \LigneParacol{0cm}
    {Hanc ígitur oblatiónem servitútis nostræ, sed et cunctæ famíliæ tuæ, quǽsumus, Dómine, ut placátus accípias~: diésque nostros in tua pace dispónas, atque ab ætérna damnatióne nos éripi et in electórum tuórum iúbeas grege numerári. Per Christum Dóminum nostrum. Amen.}
    {Voici l'offrande que nous présentons devant toi, nous, tes serviteurs, et ta famille entière. Dans ta bienveillance, accepte-la. Assure toi-même la paix de notre vie, arrache-nous à la damnation et reçois-nous parmi tes élus.}

    \LigneParacol{0cm}
    {}{\Rubrique{(On se met à genoux.)}}

    \LigneParacol{0cm}
    {Quam oblatiónem tu, Deus, in ómnibus, quǽsumus, benedíctam, adscríptam, ratam, rationábilem, acceptabilémque fácere dignéris~: ut nobis Corpus et Sanguis fiat dilectíssimi Fílii tui, Dómini nostri Iesu Christi.}
    {Sanctifie pleinement cette offrande par la puissance de ta bénédiction, rends-la parfaite et digne de toi~: qu'elle devienne pour nous le corps et le sang de ton Fils bien-aimé, Jésus Christ, notre Seigneur.}

    \LigneParacol{0cm}
    {Qui, prídie quam paterétur, accépit panem in sanctas ac venerábiles manus suas, et elevátis óculis in cælum ad te Deum Patrem suum omnipoténtem, tibi grátias agens, benedíxit, fregit, dedítque discípulis suis, dicens~:\vspace{0.3cm}}
    {La veille de sa passion, il prit le pain dans ses mains très saintes et, les yeux levés au ciel, vers toi, Dieu, son Père tout-puissant, en te rendant grâce il le bénit, le rompit, et le donna à ses disciples, en disant~:\vspace{0.3cm}}

    \LigneParacol{0cm}
    {ACCÍPITE ET MANDUCÁTE EX HOC OMNES~: HOC EST ENIM CORPUS MEUM, QUOD PRO VOBIS TRADÉTUR\vspace{0.3cm}}
    {PRENEZ, ET MANGEZ-EN TOUS, CAR CECI EST MON CORPS, LIVRÉ POUR VOUS.\vspace{0.3cm}}

    \LigneParacol{0cm}
    {Símili modo, postquam cenátum est, accípiens et hunc præclárum cálicem in sanctas ac venerábiles manus suas, item tibi grátias agens benedíxit, dedítque discípulis suis, dicens~:\vspace{0.3cm}}
    {De même, à la fin du repas, Il prit dans ses mains cette coupe incomparable~; et te rendant grâce à nouveau il la bénit, et la donna a ses disciples, en disant~:\vspace{0.3cm}}

    \LigneParacol{0cm}
    {ACCÍPITE ET BÍBITE EX EO OMNES~: HIC EST ENIM CALIX SÁNGUINIS MEI, NOVI ET ÆTÉRNI TESTAMÉNTI, QUI PRO VOBIS ET PRO MULTIS EFFUNDÉTUR IN REMISSIÓNEM PECCATÓRUM. HOC FÁCITE IN MEAM COMMEMORATIÓNEM.\vspace{0.3cm}}
    {PRENEZ, ET BUVEZ-EN TOUS, CAR CECI EST LA COUPE DE MON SANG, LE SANG DE L'ALLIANCE NOUVELLE ET ÉTERNELLE, QUI SERA VERSÉ POUR VOUS ET POUR LA MULTITUDE EN RÉMISSION DES PÉCHÉS. VOUS FEREZ CELA EN MÉMOIRE DE MOI.\vspace{0.3cm}}

\end{paracol}

\PartocheWithTraduction{GR/ordinaire/mysterium_fidei}

\begin{paracol}{2}

    \LigneParacol{0cm}
    {Unde et mémores, Dómine, nos servi tui, sed et plebs tua sancta, eiúsdem Christi, Fílii tui Dómini nostri, tam beátæ passiónis, necnon et ab ínferis resurrectiónis, sed et in cælos gloriósæ ascensiónis~: offérimus præcláræ maiestáti tuæ, de tuis donis ac datis, hóstiam puram, hóstiam sanctam, hóstiam immaculátam, panem sanctum vitæ ætérnæ et cálicem salútis perpétuæ.}
    {C'est pourquoi nous aussi, tes serviteurs, et ton peuple saint avec nous, faisant mémoire de la passion bienheureuse de ton Fils, Jésus Christ, notre Seigneur, de sa résurrection du séjour des morts et de sa glorieuse ascension dans le ciel, nous te présentons, Dieu de gloire et de majesté, cette offrande prélevée sur les biens que tu nous donnes, le sacrifice pur et saint, le sacrifice parfait, pain de la vie éternelle et coupe du salut.}

    \LigneParacol{0cm}
    {Supra quæ propítio ac seréno vultu respícere dignéris~: et accépta habére, sícuti accépta habére dignátus es múnera púeri tui iusti Abel, et sacrifícium Patriárchæ nostri Abrahæ, et quod tibi óbtulit summus sacérdos tuus Melchísedech, sanctum sacrifícium, immaculátam hóstiam.}
    {Et comme il t'a plu d'accueillir les présents d'Abel le Juste, le sacrifice de notre père Abraham, et celui que t'offrit Melchisédech, ton grand prêtre, en signe du sacrifice parfait, regarde cette offrande avec amour et, dans ta bienveillance, accepte-la.}

    \LigneParacol{0cm}
    {Súpplices te rogámus, omnípotens De\-us~: iube hæc perférri per manus sancti Angeli tui in sublíme altáre tuum, in conspéctu divínæ maiestátis tuæ~; ut, quotquot ex hac altáris participatióne sacrosánctum Fílii tui Corpus et Sánguinem sumpsérimus, omni benedictióne cælésti et grátia repleámur. Per Christum Dóminum nostrum. Amen.}
    {Nous t'en supplions, Dieu tout-puis\-sant~: qu'elle soit portée par ton ange en présence de ta gloire, sur ton autel céleste, afin qu'en recevant ici, par notre communion à l'autel, le corps et le sang de ton Fils, nous soyons comblés de ta grâce et de tes bénédictions.}

    \LigneParacol{0cm}
    {Meménto étiam, Dómine, famulórum famularúmque tuárum → et →, qui nos præcessérunt cum signo fídei, et dórmiunt in somno pacis. Ipsis, Dómine, et ómnibus in Christo quiescéntibus, locum refrigérii, lucis et pacis, ut indúlgeas, deprecámur. Per Christum Dóminum nostrum. Amen.}
    {Souviens-toi de tes serviteurs (de → et →) qui nous ont précédés, marqués du signe de la foi, et qui dorment dans la paix… Pour eux et pour tous ceux qui reposent dans le Christ, nous implorons ta bonté~: qu'ils entrent dans la joie, la paix et la lumière.}

    \LigneParacol{0cm}
    {Nobis quoque peccatóribus fámulis tuis, de multitúdine miseratiónum tuárum sperántibus, partem áliquam et societátem donáre dignéris cum tuis sanctis Apóstolis et Martýribus~: cum Ioánne, Stéphano, Matthía, Bárnaba, Ignátio, Alexándro, Marcéllino, Petro, Felicitáte, Perpétua, Agatha, Lúcia, Agnéte, Cæcília, Anastásia et ómnibus Sanctis tuis~: intra quorum nos consórtium, non æstimátor mériti, sed véniæ, quǽsumus, largítor admítte. Per Christum Dóminum nostrum, per quem hæc ómnia, Dómine, semper bona creas, sanctíficas, vivíficas, benedícis, et præstas nobis.}
    {Et nous, pécheurs, qui mettons notre espérance en ta miséricorde inépuisable, admets-nous dans la communauté des bienheureux Apôtres et Martyrs, de Jean-Baptiste, Étienne, Matthias et Barnabé, Ignace, Alexandre, Marcellin et Pierre, Félicité et Perpétue, Agathe, Lucie, Agnès, Cécile, Anastasie, et de tous les saints. Accueille-nous dans leur compagnie, sans nous juger sur le mérite mais en accordant ton pardon, par Jésus Christ, notre Seigneur. C'est par lui que tu ne cesses de créer tous ces biens, que tu les bénis, leur donnes la vie, les sanctifies et nous en fais le don.}

\end{paracol}

\PartocheWithTraduction{GR/ordinaire/per_ipsum}
