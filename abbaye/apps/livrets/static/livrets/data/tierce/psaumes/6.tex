\TitreC{Divisions du Psaume 17(18)}

\begin{paracol}{2}

\LigneParacol{0cm}
{Quóniam quis Deus præter Dóminum~? \GreStar{} aut quis Deus præter Deum nostrum~?}
{Car qui est Dieu, si ce n'est le Seigneur et qui est Dieu, si ce n'est notre Dieu~? }

\LigneParacol{0.2cm}
{Deus, qui præcínxit me virtúte~: \GreStar{} et pósuit immaculátam viam meam.}
{Le Dieu qui me ceint de force, qui rend ma voie parfaite~; }

\LigneParacol{0.2cm}
{Qui perfécit pedes meos tamquam cervórum, \GreStar{} et super excélsa státuens me.}
{qui rend mes pieds semblables à ceux des biches, et me fait tenir debout sur mes hauteurs~; }

\LigneParacol{0.2cm}
{Qui docet manus meas ad prǽlium~: \GreStar{} et posuísti, ut arcum ǽreum, brácchia mea.}
{qui forme mes mains au combat, et mes bras tendent l'arc d'airain. }

\LigneParacol{0.2cm}
{Et dedísti mihi protectiónem salútis tuæ~: \GreStar{} et déxtera tua suscépit me~:}
{Tu m'as donné le bouclier de ton salut, et ta droite me soutient,}

\LigneParacol{0.2cm}
{Et disciplína tua corréxit me in finem~: \GreStar{} et disciplína tua ipsa me docébit.}
{et ta douceur me fait grandir. }

\LigneParacol{0.2cm}
{Dilatásti gressus meos subtus me~: \GreStar{} et non sunt infirmáta vestígia mea~:}
{Tu élargis mon pas au-dessous de moi, et mes pieds ne chancellent point. }

\LigneParacol{0.2cm}
{Pérsequar inimícos meos et comprehéndam illos~: \GreStar{} et non convértar, donec defíciant.}
{Je poursuis mes ennemis et je les atteins~; je ne reviens pas sans les avoir anéantis. }

\LigneParacol{0.2cm}
{Confríngam illos, nec póterunt stare~: \GreStar{} cadent subtus pedes meos.}
{Je les brise, et ils ne se relèvent pas~; Ils tombent sous mes pieds. }

\end{paracol}
\Gloria
\begin{paracol}{2}

\LigneParacol{0cm}
{Et præcinxísti me virtúte ad bellum~: \GreStar{} et supplantásti insurgéntes in me subtus me.}
{Tu me ceins de force pour le combat, tu fais plier sous moi mes adversaires. }

\LigneParacol{0.2cm}
{Et inimícos meos dedísti mihi dorsum, \GreStar{} et odiéntes me disperdidísti.}
{Mes ennemis~!… tu leur fais tourner le dos devant moi~; et j'extermine ceux qui me haïssent. }

\LigneParacol{0.2cm}
{Clamavérunt, nec erat qui salvos fáceret ad Dóminum~: \GreStar{} nec exaudívit eos.}
{Ils crient, et personne pour les sauver~! Ils crient vers le Seigneur, et il ne leur répond pas~! }

\LigneParacol{0.2cm}
{Et commínuam illos, ut púlverem ante fáciem venti~: \GreStar{} ut lutum plateárum delébo eos.}
{Je les broie comme la poussière livrée au vent, je les balaie comme la boue des rues. }

\LigneParacol{0.2cm}
{Erípies me de contradictiónibus pópuli~: \GreStar{} constítues me in caput géntium.}
{Tu me délivres des révoltes du peuple, tu me mets à la tête des nations~;}

\LigneParacol{0.2cm}
{Pópulus quem non cognóvi servívit mihi~: \GreStar{} in audítu auris obedívit mihi.}
{Des peuples que je ne connaissais pas me sont asservis. Dès qu'ils ont entendu, ils m'obéissent~;}

\LigneParacol{0.2cm}
{Fílii aliéni mentíti sunt mihi, Þ fílii aliéni inveteráti sunt, \GreStar{} et claudicavérunt a sémitis suis.}
{les fils de l'étranger me flattent. Les fils de l'étranger sont défaillants, ils sortent tremblants de leurs forteresses. }

\end{paracol}
\Gloria
\begin{paracol}{2}

\LigneParacol{0cm}
{Vivit Dóminus, et benedíctus Deus meus~: \GreStar{} et exaltétur Deus salútis meæ.}
{Vive le Seigneur et béni soit mon rocher~! Que le Dieu de mon salut soit exalté~; }

\LigneParacol{0.2cm}
{Deus, qui das vindíctas mihi, et subdis pópulos sub me~: \GreStar{} liberátor meus de inimícis meis iracúndis.}
{Dieu qui m'accorde des vengeances, qui me soumet les peuples, qui me délivre de mes ennemis~!}

\LigneParacol{0.2cm}
{Et ab insurgéntibus in me exaltábis me~: \GreStar{} a viro iníquo erípies me.}
{Oui, tu m'élèves au-dessus de mes adversaires, tu me sauves de l'homme de violence. }

\LigneParacol{0.2cm}
{Proptérea confitébor tibi in natiónibus, Dómine~: \GreStar{} et nómini tuo psalmum dicam.}
{C'est pourquoi je te louerai parmi les nations, ô Seigneur~; je chanterai à la gloire de ton nom~:}

\LigneParacol{0.2cm}
{Magníficans salútes Regis ejus, Þ et fáciens misericórdiam Christo suo David~: \GreStar{} et sémini ejus usque in sǽculum.}
{Il accorde de glorieuses délivrances à son roi, il fait miséricorde à son oint, à David et à sa postérité pour toujours. }

\end{paracol}
