\TitreC{Divisions du Psaume 118(119)}

\begin{paracol}{2}

\LigneParacol{0cm}
{Legem pone mihi, Dómine, viam justificatiónum tuárum~: \GreStar{} et exquíram eam semper.}
{Enseigne-moi, Seigneur, la voie de tes préceptes, afin que je la garde jusqu'à la fin de ma vie.}

\LigneParacol{0.2cm}
{Da mihi intelléctum, et scrutábor legem tuam~: \GreStar{} et custódiam illam in toto corde meo.}
{Donne-moi l'intelligence pour que je garde ta loi, et que je l'observe de tout mon coeur.}

\LigneParacol{0.2cm}
{Deduc me in sémitam mandatórum tuórum~: \GreStar{} quia ipsam vólui.}
{Conduis-moi dans le sentier de tes commandements, car j'y trouve le bonheur.}

\LigneParacol{0.2cm}
{Inclína cor meum in testimónia tua~: \GreStar{} et non in avarítiam.}
{Incline mon coeur vers tes enseignements, et non vers le gain.}

\LigneParacol{0.2cm}
{Avérte óculos meos ne vídeant vanitátem~: \GreStar{} in via tua vivífica me.}
{Détourne mes yeux pour qu'ils ne voient point la vanité, fais-moi vivre dans ta voie.}

\LigneParacol{0.2cm}
{Státue servo tuo elóquium tuum, \GreStar{} in timóre tuo.}
{Accomplis envers ton serviteur ta promesse, que tu as faite à ceux qui te craignent.}

\LigneParacol{0.2cm}
{Ámputa oppróbrium meum quod suspicátus sum~: \GreStar{} quia judícia tua jucúnda.}
{Ecarte de moi l'opprobre que je redoute, car tes préceptes sont bons.}

\LigneParacol{0.2cm}
{Ecce, concupívi mandáta tua~: \GreStar{} in æquitáte tua vivífica me.}
{Je désire ardemment pratiquer tes ordonnances~: par ta justice, fais-moi vivre. }

\end{paracol}
\Gloria
\begin{paracol}{2}

\LigneParacol{0cm}
{Et véniat super me misericórdia tua, Dómine~: \GreStar{} salutáre tuum secúndum elóquium tuum.}
{Que vienne sur moi ta miséricorde, Seigneur, et ton salut, selon ta parole~!}

\LigneParacol{0.2cm}
{Et respondébo exprobrántibus mihi verbum~: \GreStar{} quia sperávi in sermónibus tuis.}
{Et je pourrai répondre à celui qui m'outrage, car je me confie en ta parole.}

\LigneParacol{0.2cm}
{Et ne áuferas de ore meo verbum veritátis usquequáque~: \GreStar{} quia in judíciis tuis supersperávi.}
{N'ôte pas entièrement de ma bouche la parole de vérité, car j'espère en tes préceptes.}

\LigneParacol{0.2cm}
{Et custódiam legem tuam semper~: \GreStar{} in sǽculum et in sǽculum sǽculi.}
{Je veux garder ta loi constamment, toujours et à perpétuité.}

\LigneParacol{0.2cm}
{Et ambulábam in latitúdine~: \GreStar{} quia mandáta tua exquisívi.}
{Je marcherai au large, car je recherche tes ordonnances.}

\LigneParacol{0.2cm}
{Et loquébar in testimóniis tuis in conspéctu regum~: \GreStar{} et non confundébar.}
{Je parlerai de tes enseignements devant les rois, et je n'aurai point de honte.}

\LigneParacol{0.2cm}
{Et meditábar in mandátis tuis, \GreStar{} quæ diléxi.}
{Je ferai mes délices de tes commandements, car je les aime.}

\LigneParacol{0.2cm}
{Et levávi manus meas ad mandáta tua, quæ diléxi~: \GreStar{} et exercébar in justificatiónibus tuis.}
{J'élèverai mes mains vers tes commandements que j'aime, et je méditerai tes lois. }

\end{paracol}
\Gloria
\begin{paracol}{2}

\LigneParacol{0cm}
{Memor esto verbi tui servo tuo, \GreStar{} in quo mihi spem dedísti.}
{Souviens-toi de la parole donnée à ton serviteur, sur laquelle tu fais reposer mon espérance.}

\LigneParacol{0.2cm}
{Hæc me consoláta est in humilitáte mea~: \GreStar{} quia elóquium tuum vivificávit me.}
{C'est ma consolation dans la misère, que ta parole me rende la vie.}

\LigneParacol{0.2cm}
{Supérbi iníque agébant usquequáque~: \GreStar{} a lege autem tua non declinávi.}
{Des orgueilleux me prodiguent leurs railleries~: je ne m'écarte point de ta loi.}

\LigneParacol{0.2cm}
{Memor fui judiciórum tuórum a sǽculo, Dómine~: \GreStar{} et consolátus sum.}
{Je pense à tes préceptes des temps passés, Seigneur, et je me console}

\LigneParacol{0.2cm}
{Deféctio ténuit me, \GreStar{} pro peccatóribus derelinquéntibus legem tuam.}
{L'indignation me saisit à cause des méchants, qui abandonnent ta loi.}

\LigneParacol{0.2cm}
{Cantábiles mihi erant justificatiónes tuæ, \GreStar{} in loco peregrinatiónis meæ.}
{Tes lois sont le sujet de mes cantiques, dans le lieu de mon pèlerinage.}

\LigneParacol{0.2cm}
{Memor fui nocte nóminis tui, Dómine~: \GreStar{} et custodívi legem tuam.}
{La nuit je me rappelle ton nom, Seigneur, et j'observe ta loi.}

\LigneParacol{0.2cm}
{Hæc facta est mihi~: \GreStar{} quia justificatiónes tuas exquisívi.}
{Voici la part qui m'est donnée~: je garde tes ordonnances.}

\end{paracol}
